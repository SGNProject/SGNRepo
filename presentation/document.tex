\documentclass{beamer}
\label{key}\usetheme{Singapore}
\usepackage{ dsfont }
%\usepackage{ tipa }

\title{Nonlinear Observer for Tightly Coupled Integration of Pseudorange and Inertial Measurements}
\subtitle{Guide and Navigation Systems}
\author{P. Bramante, L. Bertoni, F. Di Luzio}
\institute{Universit\`a degli Studi di Pisa \\ Master's Degree in Robotics and Automation Engineering}
\date{\today}

\begin{document}
	
	\begin{frame}
	\titlepage
	\end{frame}	


	\begin{frame}
		\frametitle{Outline}
		
		\begin{enumerate}
			\item Introduction
			\item Models
				\begin{enumerate}
					\item Vehicle Kinematics
					\item Inertial Sensor Models
					\item Pseudorange Measurement Model
				\end{enumerate}
			\item Observer Desing
				\begin{enumerate}
					\item Attitude Observer
					\item Translation Motion Observer
					\item Stability and Selection of Gain Matrix
				\end{enumerate}
			\item Implementation Design
			\item Case Study
		\end{enumerate}	
	\end{frame}

	\begin{frame}
		\frametitle{Introduction}
		This presentation shows the obtained results by the implementation of the article written by Tor. A. Johansen and Thor I. Fossen.
		\vspace{0.5cm}
		
	\begin{figure}[H]
			\centering
			\includegraphics[scale=0.3]{title}
			%\caption{Block diagram of the closed-loop system for $\mu$ analysis and synthesis. \label{fig::title}}
		\end{figure}
	\end{frame}

	\begin{frame}
		\frametitle{Introduction}
	%	The work has been developed in Matlab and Simulink (v. R2016b).
	The goal of navigation systems is to estimate the vehicle position within a certain area. This is let generally by a inertial navigation system, based on $IMU$ (Inertial Measurement Unit).\\
	
	Using an interconnection of a nonlinear attitude observer and a translational motion observer based on pseudorange and range-range measurements, a tightly coupled integrated aided inertial navigation system is designed!?!?!?\\
	
	Due to the problem of measurements integration, this method is supported by other techniques. In this work, the inertial navigation system is aid by pseudorange measurements, obtained by transponders.
	%The article shows hot to build an inertial navigation  system  with the aid of pseudorange measurements, obtained by proper transponders.
	\end{frame}


	\begin{frame}
	\frametitle{Vehicle Kinematics}
	The vehicle kinematic model is given by
	
	\[ \dot{p^n} = v^n \]
	
	\[ \dot{v^n} = R^n_b f^b + g^n\]
	
	\[ \dot{R^n_b} = R^n_bS(\omega^b_{ib}) \]
	
	Where $p^n$, $v^n$, $f^n$ are position, velocity and proper acceleration in NED (North-East-Down), respectively, while the attitude is described by a rotation matrix $R^n_b$ that represents the rotation from \textit{body} to NED; $\omega^b_{ib}$ represents the rotation rate of body with respect to ECI (Earth-Centered-Inertial) and $g^n$ denotes the gravity vector. We also assume NED to be an inertial frame. 
	
	\end{frame}
	
	\begin{frame}
	\frametitle{Inertial Sensor Models}
	The inertial sensor model is based on the strapdown assumption
	
	\[ f^b_{IMU} = f^b + \epsilon_f\]
	
	\[ \omega^b_{ib,IMU} = \omega^b_{ib} + b + \epsilon_\omega \]
	
	\[ \dot{b} = \epsilon_b \]
	
	\[ m^b_{mag} = m^b + \epsilon_m \]
	
	where $\epsilon_f$, $\epsilon_\omega$ and $\epsilon_m$ account for noise, and $b$ denotes the rate gyro bias that is driven by the noise $\epsilon_b$ and assumed to be bounded.  
	\\ All sensors are 3-D.
	\end{frame}
	
	\begin{frame}
	\frametitle{Pseudorange Measurement Model}
	
	The geometric range 
	
	\[ \rho_i = \|p^n - p^n_i\|_2 \]
	
	is a nonlinear function of the vehicle position $p^n$ and the $i$th transponder position $p^n_i$, given by their Euclidean distance.
	\\ The pseudorange measurement model is
	
	\[ y_i = \rho_i + \beta + \epsilon_{yi} \]
	
	where $\beta \in \mathds{R}$ is a bias parameter due to unknown clock synchronization errors or other unknown effects and $\epsilon_{yi}$ the noise.
	\\ $i = 1,2,...,m$ where $m$ is the number of measurements.
	\end{frame}
	
	\begin{frame}
	\frametitle{Pseudorange Measurement Model}
	The nonlinear model can be approximated with a linear one by an algebraic transformation 
	
		\[ 2C_{\delta x}x = \delta + \varepsilon  \]
		
	where the matrix $C_{\delta x} \in \mathds{R}^{(m-1)\times4}$ is
	
	$$
	C_{\delta x} :=
	\begin{pmatrix}
	(p^n_m - p^n_1)^T & y_1 - y_m \\
	\vdots \\
	(p^n_m - p^n_{m-1})^T & y_{m-1} - y_m 
	\end{pmatrix} 
	$$
	
	$\varepsilon \in \mathds{R}^{m-1}$ the noise and $\delta \in \mathds{R}^{m-1}$ the vector of squared range measurements. \\ $x := (p^n_\Delta;\beta)$ where $p_\Delta^n = p^n - p^n_0$ ($p_\Delta^n$ is a reference point in NED). 
	\end{frame}

	\begin{frame}
	\frametitle{Observer Design}
	Two observer are designed: one for the attitude estimation and one for the translational motion estimation. 
		\begin{figure}[H]
		\centering
		\includegraphics[scale=0.3]{observers}
		\caption{Overall block diagram for tightly integrated observer.}
	\end{figure}
	\end{frame}

	\begin{frame}
		\frametitle{Attiture Observer}
		The attitude variables to estimate are $R^n_b$ and $b$.
		
		\[ \dot{\hat{R}}^n_b  =  \hat{R}^n_b S(\omega^b_{ib,IMU} - \hat{b}) + \sigma K_pJ(t, \hat{R}^n_b)       \]
		
		\[ \dot{\hat{b}} = Proj(-k_I vex(\mathds{P}_a (sat(\hat{R}^n_b)^T K_P J(t, \hat{R}^n_b))),M_{\hat{b}} )           \]
		
		where $K_P > 0 \in \mathds{R}^{3\times 3}$ is a symmetric gain matrix, $K_I > 0$ is a scalar gain and $\sigma \geq 1$.\\ 
		
		The function $sat(\cdot)$ is an element-wise saturation, while $Proj(\cdot)$ is a parameter projection which ensures that $\|\hat{b} \|_2$ is bounded.
	\end{frame}

	\begin{frame}
 		\frametitle{Attiture Observer}
 		The function $J(\cdot) \in  \mathds{R}^{3\times 3}$ is a stabilizing injection term
 		
 		\[ J(t, \hat{R}^n_b)  = (E^n - \hat{R}^n_b E^b)(E^b)^T    \]
 		
 		based on the vector measurements $m^b_{mag}$ and $f^b_{IMU}$  and their NED reference vetors $m^n$ and $\hat{f}^n$ used to define vectors scaled by nonzero terms
 		
 		\[ q_1^b = m^b_{mag}/\|m^b_{mag}\|_2 \qquad q_2^b = f^b_{IMU}/\|g^n\|_2    \]
 		
 		\[ q_1^n = m^n/\|m^n\|_2 \qquad q_2^n = \hat{f}^n/\|g^n\|_2    \]
 		
 		and the $3\times3$ matrices
 		
 		\[ E^b = (q_1^b, S(q^b_1)q_2^b, S^2(q_1^b)q_2^b)    \]
 		
 		\[ E^n = (q_1^n, S(q^n_1)q_2^n, S^2(q_1^n)q_2^n)    \]
	\end{frame}


	\begin{frame}
		\frametitle{Translational Motion Observer}
		The variables to estimate are $p^n$,$v^n$,$f^n$,$\beta$.
		
		\[ \dot{\hat{p}}^n_\Delta = \hat{v}^n + K_{pp}(\delta - \hat{\delta})  \]
		
		\[ \dot{\hat{\beta}} = K_{\beta p} (\delta - \hat{\delta}) \]
		
		\[ \dot{\hat{v}}^n = \hat{f}^n + g^n + K_{vp}(\delta - \hat{\delta})\]
		
		\[ \dot{\xi} = - \sigma K_P J(t, \hat{R}^n_b)f^b_{IMU} + K_{\xi p}(\delta - \hat{\delta}) \]
		
		\[ \hat{f}^n = \hat{R}^n_b f^b_{IMU} + \xi  \]
		
		where $\hat{\delta} = 2C_{\delta x}\hat{x}$ and the gain matrix $K \in \mathds{R}^{10\times(m-1)}$ is made of the matrices $K_*$ and is in general time varying.
	\end{frame}

	\begin{frame}
		\frametitle{Translational Motion Observer}
		Then it is possible to build the \textit{estimated state} vector $\dot{\tilde{\chi}} = (\tilde{p}^n_\Delta;\tilde{\beta}; \tilde{\hat{v}}^n; \tilde{\hat{f}}^n) \in \mathds{R}^{10}$ and the relative LTV error system
		
		\[ \dot{\tilde{\chi}} = (A - KC)\tilde{\chi} + Bu + B\epsilon_u + K\varepsilon\]
		
		\[ u = \tilde{R}^n_b \dot{f}^b + \tilde{R}^n_b S(\omega^b_{ib})f^b - \hat{R}^n_b S(\tilde{b})f^b\]
	\end{frame}

	\begin{frame}
		\frametitle{Translational Motion Observer}
		The matrices $A \in \mathbb{R}^{10\times 10}$,$B \in \mathbb{R}^{10\times 3}$,$C \in \mathds{R}^{(m-1) \times 10} $ and $K \in \mathds{R}^{10 \times (m-1)}$ are described as follows
		$$
		A :=
		\begin{pmatrix}
		0 & 0 & I_3 & 0 \\ 
		0 & 0 & 0 & 0 \\
		0 & 0 & 0 & I_3 \\
		0 & 0 & 0 & 0
		\end{pmatrix}
		\qquad
		B := 
		\begin{pmatrix}
		0 \\ 0 \\ 0 \\ I_3
		\end{pmatrix}
		$$
		
		$$
		K := 
		\begin{pmatrix}
		K_{pp} \\ K_{\beta p} \\ K_{vp} \\ K_{\xi p}
		\end{pmatrix}
		\qquad
		C :=
		\begin{pmatrix}
		2C_{\delta x} & 0 & 0
		\end{pmatrix}
		$$
	\end{frame}

	\begin{frame}
		\frametitle{Translational Motion Observer}
		The gain matrix $K$ is time varying and calculated as
		
		\[ K := PC^TR^{-1} \]
		
		 where $P$ is solution of the $Riccati$ equation
		
		\[ \dot{P} = PA + A^TP - PC^TR^{-1}CP + Q\]
	\end{frame}
	
	\begin{frame}
		\frametitle{Implementation Design}
		The whole model has been implemented with MATLAB/Simulink (R2016b).
		\\
		\vspace{0.5cm}
		In the following each observer is presented in terms of structure and results, i.e. the difference between the true value of the vehicle position and of the bias parameter and their estimation.
		
	\end{frame}
	
	\begin{frame}
		\frametitle{Implementation Design}
		Data used by the authors are \\
		\begin{itemize}
			\item pseudorange measurements obtained by radio beacons;
			\item accelerations and angular velocities by IMU. 
		\end{itemize}
		\vspace{0.5cm}
		The inertial sensor model is based on the strapdown assumption, i.e. the inertial measurement unit is fixed to the body frame.
	\end{frame}
	
	\begin{frame}
		\frametitle{Data Generation}
		In order to generate data the $6DOF$ Matlab block has been used
		\begin{figure}[H]
			\includegraphics[scale=0.4]{6DOF}
		\end{figure}
	\end{frame}
	
	\begin{frame}
		\frametitle{Data Generation}
		\begin{itemize}
			\item The real position of the vehicle is given from $X_e(m)$;
			\item The pseudorange measurements always by $X_e(m)$ but corrupted by noise;
			\item The accelerations from $A_b(m/s^2)$;
			\item Angular velocities from $\omega_b(rad/s)$.
		\end{itemize}
	\end{frame}
	
	\begin{frame}
		\frametitle{Implementation Design}
		The Simulink implementation looks like the one in the following figure.
		\begin{figure}[H]
			\includegraphics[scale = 0.3]{scheme}
		\end{figure}
	\end{frame}
	
	\begin{frame}
			\frametitle{Implementation Design}
			The $Nonlinear Algebraic Transform$ block computes the $C_{\delta x}$ matrix and the following variable:
			
			\[\hat{x} = \frac {C_{\delta x}^{+} \delta}{2}  \]
			
			that is the unique solution to
			
			\[ 2C_{\delta x} x = \delta + \epsilon \]
			
			in the case of $m \geqslant 5$ transponders, where $x = \begin{pmatrix}
				p^n \\
				\beta
			\end{pmatrix}$.
		\end{frame}
		
		\begin{frame}
			\frametitle{Nonlinear Algebraic Transform Implementation}
			\includegraphics[scale = 0.24]{NLAT}
		\end{frame}
		
		\begin{frame}
			figures
		\end{frame}
		
		
	\begin{frame}
		\frametitle{Implementation Design}
		The Attitude Observer is made of three main function blocks:\\
		\begin{itemize}
			\item $J(\cdot) $ is a stabilizing injection term
			
			\item $b\_computation$ computes the dinamics of the bias by means of the $Proj(\cdot)$ function
			
			\item $Rot\_func$ computes an estimate of the rotation matrix
		\end{itemize}
		
	\end{frame}
	
	\begin{frame}
		The following picture shows the Simulink scheme:
		\frametitle{Attitude Observer Implementation}
		\includegraphics[scale=0.24]{at}
	\end{frame}

	\begin{frame}
		\frametitle{Translational Motion Observer Design}
		The translational Motion Observer is made of three blocks:\\
		
		\begin{itemize}
			\item The first one solves the Riccati Equation to get the gain matrix $K$;
			
			\item The $TMO\_fuction$ computes the estimate of position, velocity, acceleration and bias;
			
			\item The last block computes the error dynamics.
		\end{itemize}
	\end{frame}
	
	\begin{frame}
		\frametitle{Translational Motion Observer Design}
		\includegraphics[scale = 0.25]{TMO}
	\end{frame}
	
	\begin{frame}
		figures
	\end{frame}
	
	\begin{frame}
		\frametitle{Implementation Design}
		The estimates obtained by the Translational Motion Observer are then sent to a block that implements the Kalman-Bucy Filter in order to make a better estimation
		
		\begin{figure}[H]
			\includegraphics[scale=0.3]{KBF}
		\end{figure}
		
	\end{frame}
	
	\begin{frame}
		Kalman Equations
	\end{frame}
	
	\begin{frame}
			figures
	\end{frame}
	
	\begin{frame}
			Confronto tra gli osservatori
	\end{frame}
\end{document}